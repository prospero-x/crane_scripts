\documentclass[twocolumn,10pt,a4paper]{article}
\usepackage{amsmath}
\usepackage{amssymb}
\usepackage[english]{babel}
\usepackage{calrsfs}
\usepackage{chngcntr}
\usepackage{comment}
\usepackage{dsfont}
\usepackage{esint}
\usepackage{fancyhdr}
\usepackage[T1]{fontenc}
\usepackage{gensymb}
\usepackage{geometry}
\usepackage{graphicx}
\usepackage[utf8]{inputenc}
\usepackage{lipsum}
\usepackage{lmodern}
\usepackage{mathtools}
\usepackage[linewidth=1pt]{mdframed}
\usepackage{multicol}
\usepackage[section]{placeins}
\usepackage{pgfplots}
\usepackage{pgfplotstable}
\usepgfplotslibrary{external}
\usepackage{physics}
\usepackage{subcaption}
\usepackage{tcolorbox}
\usepackage{upgreek}
\usepackage{wrapfig}
\usepackage{yfonts}

%
%
% Page margins
%
%
\geometry{
    margin = 1cm,
    top = 1in,
    bottom = 1in,
}

%
%
% Fancy headers and footers
%
%
\pagestyle{fancy}
\fancyhf{}
\fancyhead[LE,LO]{NPRE 598 Module 3: Fluid Models}
\fancyhead[RE,RO]{Instructors: Davide Curreli, Shane Keniley}
\fancyfoot[LE,LO]{Module 3: Fluid Models}
\fancyfoot[CE,CO]{\thepage}




% Equations take the same number as the section they're in
\counterwithin{equation}{section}

\title{
        CO2 Splitting in a Microwave Plasma
}
\author{Mikhail Rezazadeh}% Instructors: Davide Curreli, Shane Keniley}
\date{}
\begin{document}
\twocolumn [
        \begin{@twocolumnfalse}
        \maketitle

        \begin{abstract}
                As atmospheric CO$_2$ becomes an increasing problem, several
                methods for CO$_2$ mitigation have been suggested. Recently,
                the use of plasma technology to split CO$_2$ has gained
                attention within the scientific community. An important step
                for demonstrating the feasibility of plasma technology is
                maximizing the efficiency of the splitting process. This paper
                examines the effects of gas temperature on the splitting of
                CO$_2$ in a Microwave Plasma.
        \end{abstract}
\vspace{1cm}
\end{@twocolumnfalse}
]


%
%
% Introduction
%
%
\section{Introduction}
As the concentration of CO$_2$, which is a greenhouse gas, increases in Earth's atmosphere,
several methods for mitigation have been proposed. One method gaining attention is the
splitting of CO$_2$ in a supersonic flow in a microwave plasma discharge. Recently, the efficiency
has reached as high as 90\% for such systems. In order to better understand these complex systems,
computer simulation is needed. In this paper, I show the effects of altering one parameter, gas
temperature, on the percent dissociation of CO$_2$ in a Microwave plasma of constant electon
temperature of 2eV.


\section{Model Description}

\subsection{Plasma Conditions}
For this model, a constant electron temperature of 2eV was used.  The mean
electron energy is calculated using the formula $\langle E \rangle = 3/2 k_BT$. The pressure
remained constant at 100 mBar. For electron
impact reactions, cross sections were obtained from LxCat, specifically the Phelps
and Itikawa databases.

\subsection{Bolsig+}
Electron impact reaction rates were calculated using Bolsig+.
For the Bolsig input conditions, the plasma density was required.
This was estimated as a function of gas temperature by examining Figure 4 in
[1] (sub plot (d), with power deposition equal to Pmax*5 and pressure of 100
mBar). For example, at a temperature of 300K, a plasma density of $2 \cross 10^{14}$
m$^{-3}$ was used, and for a temperature of 3000K, the plasma density was estimated to be
$10 \cross 10^{17}$  m$^{-3}$. Since vibrational temperature in Figure 4 roughly
approximated the gas temperature, the gas temperature was used as the Excitation
Temperature in the Bolsig+ configuration.

\subsection{Crane Configuration}
After electron impact reaction rates were calculated using Bolsig+, CRANE
(developed by the Laboratory for Computational Plasma Physics at UIUC) was used
to solve the 0-Dimensional reaction network problem. Crane was configured with a starting
CO2 density by using the Ideal Gas Law for CO2 at 100 mBar. The initial electron
density was set using the same process as above, namely referring to Figure 4
of [1].


\subsection{Reduction of Reaction Network}
The reaction network in the supplementary information of (Boegaerts and Bethelot 2017))
was used to simulate
a surfatron microwave steady-state discharge of a CO$_2$ plasma. To reduce the reaction,
the well-known Quasi Steady State Assumption (QSSA) was
applied, distinguishing between so-called fast and slow reactions. A threshold
reaction rate of $1 \cross 10^{-16}$ (m$^{3}$/s or m$^{6}$/s) was chosen, and
every reaction with a constant reaction rate (i.e., independent of EEDF,
electron temperature, and gas temperature) less than this was removed from
the reaction network.

Next, reaction rates with non-constant rates were examined and eliminated on
a case-by-case basis. For example, the reaction:
\begin{align*}
        \text{CO} + \text{O}^+ &\longrightarrow \text{CO}^+ + \text{O}\\
        \text{Rate:   }& 4.9 \cross 10^{-18}\sqrt{\frac{T_g}{300K}}
            \text{exp}\left(\frac{-4580K}{T_g}\right)
\end{align*}
were $T_g$ is gas temperature in Kelvin, will be eliminated because the maximim
rate, which occurs at 3000K, is 3.07$\cross 10^{-18}$ m$^3$/s, which is below the
threshold. However, the following reaction:
\begin{align*}
        \text{O}_2^-   + \text{M} &\longrightarrow \text{O}_2 + \text{M} + \text{e}\\
        \text{Rate:   }& 2.7 \cross 10^{-16}\sqrt{\frac{T_g}{300K}}
            \text{exp}\left(\frac{-5590K}{T_g}\right)
\end{align*}
has a maximum rate at 3000K of 1.32 $\cross 10^{-16}$ m$^{3}$/s, which is
above the threshold. Therefore, it is included in the final reaction network.

Finally, the electron impact reactions which neither consumed nor produced CO$_2$
were removed from the reaction network, leaving a total of 11 reactions, as shown
below:

\begin{figure}[!htb]
\begin{center}
\resizebox{8cm}{!}{
 \begin{tabular}{|| c | l ||}
 \hline
         \textbf{Name} & \textbf{Electron Impact Reaction} \\
         [0.5ex]
\hline\hline
         X1 &e + CO$_2$ $\longrightarrow$ e + e + CO$_2^+$\\
         X2 &e + CO$_2$ $\longrightarrow$ e + e + CO + O$^+$\\
         X3 &e + CO$_2$ $\longrightarrow$ e + e + O + CO$^+$\\
         X4& e + CO$_2$ $\longrightarrow$ e + e + O$_2$ + C$^+$\\
         X5&e + CO$_2$ $\longrightarrow$ CO + O$^-$ \\
         X6&e + CO$_2$ $\longrightarrow$ CO + O + e \\
         X7&e + CO$_2$ $\longrightarrow$ e + CO$_2$* (10.5 eV) \\
         X8&e + CO$_2$ $\longrightarrow$ e + CO$_2$v$_1$ (0.083 eV) \\
         X9&e + CO$_2$ $\longrightarrow$ e + CO$_2$v$_2$ (0.167 eV) \\
         X10&e + CO$_2$ $\longrightarrow$ e + CO$_2$v$_3$ (0.252 eV) \\
         X11&e + CO$_2$ $\longrightarrow$ e + CO$_2$v$_4$ (0.339 eV) \\
\hline
\end{tabular}}\\
\caption{Electron Impact reactions used in reaction network. Only reactions
        involving CO$_2$ were used.}
\end{center}
\end{figure}

The reason for using only reactions involving CO$_2$ is that we wish to examine
the dissociation of CO$_2$ reactions. While other reactions might consume
or produce species which are involved in other CO$_2$ production or dissociation
reactions, we assume here that the first-order approximation involves only
CO$_2$ electron impact reactions. Note that X2, X3, and X4 were not available
on LxCat, but were taken from Itikawa 2002 ([4])

\section{Production Runs}
Crane was used to solve systems of the above reaction model for 50 temperatures,
equally spaced between 300K and 3000K. For reactions involving heavy neutrals "M", M was calculated at
each timestep as the sum of the densities of all neutral species in the plasma. The time variable was set from $t=0$ to 1 millisecond.
The MOOSE standard library \textbf{IterationAdaptiveDT} was
used, which dynamically scaled the time step up and down based
on the number of steps taken to converge to the previous step.

\section{Numerical Convergence}
All timesteps for all Crane simulations converged within 1
iteration (both Linear and Nonlinear Solves). An example,
the Steady-State Relative Differential Norms for the 2700K run
are shown in Figures~\ref{fig: co2norms} and \ref{fig: ssnorms}
\begin{figure}[htbp]
\centering
\includegraphics[width = 8cm]{plots/co2_residual_norms.png}
\caption{CO$_2$ residual norms for 2700K simulation.}
\label{fig: co2norms}
\end{figure}


\begin{figure}[htbp]
\centering
        \includegraphics[width = 8cm]{plots/steady_state_norms.png}
       \caption{Steady-State Relative Differential Norms for 2700 K simulation.}
        \label{fig: ssnorms}
\end{figure}

For each timestep, the linear and nonilnear convergence took 1 iteration, meaning that the IterativeAdaptive
time step increased exponentially for the entire simlation. While the residual norm for CO$_2$ is high,
it is on the same order of the CO$_2$ volumetric density.

\section{Results}
The results of the temperature-parametrized simulations are shown in Figure \ref{fig: pct}. Percent
conversion is defined by:
\begin{align*}
        \%\text{ converted} = \frac{
            n(\text{CO}_2)_\text{start} - n(\text{CO}_2)_\text{end}
        }
        {
            n(\text{CO}_2)_\text{start}
        }
\end{align*}

\begin{figure}[htbp]
\begin{center}
\includegraphics[width=8cm]{plots/pct_converted.png}
\caption{Percent CO$_2$ Converted in 1 millisecond of Plasma Simulation}
\label{fig: pct}
\end{center}
\end{figure}

\section{Discussion}
\subsection{Temperature Dependence of CO2 Conversion}
The results in Figure~\ref{fig: pct} demonstrate a much greater conversion rate of CO$_2$ at higher
gas temperatures. This can be explained by what Boegaerts and Berthelot (2017) refer to as the main CO$_2$ dissociation reactions:
\begin{align*}
        \text{Electron Impact Dissociation} &\text{  (X7)}\\
        \text{CO}_2 + \text{M} \longrightarrow \text{CO} + \text{O} + \text{M} &\text{  (N1)}\\
        \text{CO}_2 + \text{O} \longrightarrow \text{CO} + \text{O}_2   &\text{  (N2) }\\
\end{align*}
At low temperatures (300K) the rates of N1 and N2 are highly power-dependent.
As gas temperature increases, N1 and N2 become increasingly constant across all powers.
X1, on the other hand, increases linearly with power deposition regardless of the gas temperature.
This can be explained from the fact that electron density increases with power deposition, while
the electron temperature remains constant. Thus, the rate coefficient remains constant but the
electron-impact dissociation rate increases linearly.

\subsection{Network Reduction}
In the QSSA approach, slow reactions are eliminated from the chemical reaction network (CRN). Electron-
impact reactions are typically the fast reactions in a system, and so in the context of investigating
CO$_2$ splitting, it would be best to leave intact more electron-impact reactions.

To illustrate this point, note that Boegaerts \& Bethelot (2017) identify N1 and N2 (see section 6.1) as the main CO$_2$
dissociation reactions, while the following:
\begin{align*}
    \text{CO} + \text{O} + \text{M} \longrightarrow \text{CO}_2 + \text{M} & \text{   (N3)}\\
    \text{O}_2 + \text{CO} \longrightarrow \text{CO}_2 + \text{O} & \text{   (N4)}
\end{align*}
are the main recombination reactions. Thus, an even better method for network reduction would be to
leave in place all electron-impact reactions which either produce or consume the reactants of the 2 main
dissociation reactions (N1 and N2) and the two main recombination reactions (N3 and N4). Since neutral heavy
species \textit{M} is a reactant in these reactions, that means we must include every electron-impact reaction
which either removes a heavy neutral from the plasma or produces a new heavy neutral. Howver, that would
leave intact every single electron-impact
reaction. In the spirit of network-reduction, it did not seem appropriate to copy the entire
network of electron impact reactions.

Other work ([2]) focuses on lumping together vibrational levels of CO$_2$, while [3] does a PCA analysis
to reduce the network down to two principle components. There are other interesting network reduction
techniques, such as the one in [5] which makes use Geometric Singular Perturbation Theory to obtain
an analytic method for network reduction. However, the examples in the paper involve only a few reactions,
and applying such a method to the full network for CO$_2$ microwave plasma would be a considerable
undertaking.

\section{Conclusion}
In this project, I used Crane to solve a 0-dimentional plasma problem simulating a Microwave discharge
of a CO$_2$ plasma in a surfatron. Electron-impact reaction rates were computed for each gas temperature
using Bolsig+, with cross sections coming from LxCat or directly from [4]. The results show that gas
temperature greatly improves the rate of CO$_2$ dissociation, especially for gas temperatures above 2500K.
If microwave plasmas are to be applied at a large scale to mitigate atmospheric CO$_2$, the power required
to achieve such high gas temperatures must be considered, as power efficiency is a critical metric of the
industrial splitting process.

\section{Acknowledgements}
Special thanks for Professor Davide Curreli for his guidance and instruction on Chemical networks and Plasma
reduction. In addition, this project would not have been possible without the help of Shane Keniley.

\section{References}
[1] Berthelot, A., \&amp; Bogaerts, A. (2017). Modeling of CO2 Splitting in a Microwave Plasma: How to Improve the Conversion and Energy Efficiency. Journal of Physical Chemistry C, 121.\\
\newline
[2] Berthelot, A., \&amp; Bogaerts, A. (2016). Modeling of plasma-based conversion : Lumping of the vibrational levelsAn. Plasma Sources Science and Technology, 25, 4th ser.\\
\newline
[3] Peerenboom, K., Parente, A., Kozák, T., Bogaerts, A., \&amp; Degrez, G. (2015). Dimension reduction of non-equilibrium plasma kinetic models using principal component analysis. Plasma Sources Science and Technology.\\
\newline
[4] Itikawa, Y. (2002). Cross Sections for Electron Collisions With Carbon Dioxide. Journal of Physical and Chemical Reference Data, 31.\\
\newline
[5] Feliu, E., Kruff, N., \&amp; Walcher, S. (2020). Tikhonov–Fenichel Reduction for Parameterized Critical Manifolds with Applications to Chemical Reaction Networks. Journal of Nonlinear Science, 30.
% Ref 6 https://arxiv.org/pdf/1905.08306.pdf
\end{document}
